\section{Yλοποίηση Rust}

Για την υλοποίηση σε Rust είναι αρχικά σημαντικό να ξεκαθαρίσουμε ορισμένα
πράγματα. Ο μεταγλωττιστής της Rust παρέχει στην πραγματικότητα δύο ξεχωριστές
γλώσσες, την ασφαλής (safe) Rust και την μη ασφαλής (unsafe) Rust. Η safe Rust
όπως υπονοεί το όνομα είναι το μέρος της γλώσσας που χρησιμοποιεί στατική ανάλυση
για να αποτρέψει σύνηθες σφάλματα που αφορούν την διαχείριση μνήμης. Η unsafe Rust
ενεργοποιείται με την λέξη κλειδί \verb|unsafe| και επιτρέπει ενέργειες που ενδεχομένως
μπορούν να διακόψουν (halt) την εκτέλεση του προγράμματος, όπως να κάνουμε dereference έναν
σκέτο (raw) δείκτη. Η σύνηθες υλοποίηση της λογικής της Rust που εξηγείτε στα παρακάτω κεφάλαια
διατηρείτε και στις δύο εκδοχές της γλώσσας και όπως θα δούμε μπορούν προφανώς να συνυπάρχουν.

Σε γενικές γραμμές όμως η safe Rust είναι ένα υποσύνολο της unsafe Rust. Μάλιστα διαισθητικά
τουλάχιστον μπορούμε να πούμε πως η safe Rust είναι υποσύνολο της C. Δηλαδή έστω ότι όλα τα πιθανά
προγράμματα εκφράσιμα από μια παραδοσιασκή μηχανή Turing βρίσκονται στο σύνολο $U$, όλα τα πιθανά
προγράμματα εκφράσιμα από το specification της C $C=U$ και όλα τα πιθανά προγράμματα της safe Rust $SR \subseteq C$.
Συνεχίζοντας ισχύει ότι για την unsafe rust $UR=C=U, SR \subseteq UR$. 

Έχωντας πει αυτά γίνεται ξεκάθαρη η πολυπλοκότητα του μεταγλωτιστή. Σε σύγκριση με την C η οποία 
μπορεί να μεταφραστεί σε γλώσσα μηχανής με μόνο ένα ενδιάμεσο στάδιο η Rust χρειάζεται το λιγότερο 3
στην μόνη υλοποίηση που υπάρχει. Σε αυτά τα τέσσερα στάδια, δεν μετριούνται καν οι ενδιάμεσες αναπαραστάσης
του LLVM. Στο σχήμα \ref{fig:rust_ir} η κάθε ενδιάμεση αναπαράσταση μεταλλάζει τον πηγαίο κώδικα σε όλο και
μια πιο απλή γλώσσα που στο τέλος η μηχανή μπορεί να κατανοήσει.

\begin{enumerate}
\item HIR: Είναι ο πηγαίος κώδικας αλλά desugared, δηλαδή οντότητες όπως είναι ο χρόνος
  ζωής μεταβλητών (lifetimes) ή οι τύποι των μεταβλητών εκφράζονται πλέον ρητά. Ένα
  feature σημαντικού ενδιαφέροντος για εμάς που γίνεται desugared σε αυτό το στάδιο είναι τo
  \verb|async|.
\item THIR: Σε αυτήν την αναπαράσταση η απλοποίηση του κώδικα γίνεται πιο εμφανής. Σύνθετη
  τύποι μετατρέπονται σε πιο απλούς (primitives), τα generics μεταφράζονται στις στατικές τους
  μορφές και οι μέθοδοι μετατρέπονται σε συναρτήσεις.
\item MIR: Ο λόγος ύπαρξης του THIR είναι για να γίνεται πιο απλή η μετάφραση σε αυτό το στάδιο.
  Είναι ουσιασικά ένα Control-Flow Graph (CFG) που αναπαριστά την ροή εκτέλεσης του προγράμματος.
  Εδώ γίνεται η στατική ανάλυση απαραίτητη για τον Borrow Checker, τον εγκέφαλο του στατικού ελεγκτή
  της Rust που είναι υπεύθυνος για το specification που πρέπει να ακολουθεί το πρόγραμμα μας. Το
  specification εξηγείτε με απλό τρόπο σε παρακάτω κεφάλαιο. 
\end{enumerate}

Η τελική μετατροπή είναι του LLVM το οποίο είναι ένα third-party framework που χρησιμοποιείτε ως
παγκόσμια γλώσσα μηχανής. Είναι μια υπερβολικά απλή και μινιμαλιστική γλώσσα προγραμματισμού βασισμένη
σε SSA (Static Single Assignment) η οποία μεταφράζεται στις πιο γνωστές αρχιτεκτονικές.

\begin{figure}
    \centering
    \includegraphics[scale=0.4]{images/rust/rust_irs.png}
    \caption{Rusts Intermediate Representations.}
    \label{fig:rust_ir}
\end{figure}

\subsection{Σκελετός υλοποίησης}


\begin{lstlisting}[language=Rust]
#![no_std]
#![no_main]

use embassy_executor::Spawner;
use embassy_time::{Duration, Timer};
use esp_backtrace as _;
use esp_hal::clock::CpuClock;
use log::info;

extern crate alloc;

#[esp_hal_embassy::main]
async fn main(spawner: Spawner) {
    // generator version: 0.2.2

    let config = esp_hal::Config::default().with_cpu_clock(CpuClock::max());
    let peripherals = esp_hal::init(config);

    esp_alloc::heap_allocator!(72 * 1024);

    esp_println::logger::init_logger_from_env();

    let timer0 = esp_hal::timer::systimer::SystemTimer::new(peripherals.SYSTIMER);
    esp_hal_embassy::init(timer0.alarm0);

    info!("Embassy initialized!");

    let timer1 = esp_hal::timer::timg::TimerGroup::new(peripherals.TIMG0);
    let _init = esp_wifi::init(
        timer1.timer0,
        esp_hal::rng::Rng::new(peripherals.RNG),
        peripherals.RADIO_CLK,
    )
    .unwrap();

    // TODO: Spawn some tasks
    let _ = spawner;

    loop {
        info!("Hello world!");
        Timer::after(Duration::from_secs(1)).await;
    }

    // for inspiration have a look at the examples at https://github.com/esp-rs/esp-hal/tree/v0.23.1/examples/src/bin
  }  
  \end{lstlisting}
    
\subsection{Βασική Rust}

Η λογική είναι από τις σημαντικότερες αρχές της Rust καθώς
αποτυπώνει με αυστηρό μαθηματικό τρόπο τους υποκείμενους
μηχανισμούς του μεταγλωττιστή. Μάλιστα έχουμε ήδη αναφέρει ότι
όλες οι γλώσσες συμπίπτουν στη επιστήμη της λογικής με τον έναν
ή τον άλλον τρόπο, σε καμία περίπτωση όμως δεν είναι απαραίτητη
για την κατανόηση τους. Γιατί λοιπόν είναι εξαίρεση η Rust?
Η λέξη κλειδί είναι η αυστηρότητα της γλώσσας και η απαίτηση του
μεταγλωττιστή ένα πρόγραμμα να είναι σωστό με βάσει προορισμένους
κανόνες. Το σύστημα λογικής που χρησιμοποιείτε για να εκφράσει την
εγκυρότητα ενός προγράμματος είναι ένα υποσύνολο της γραμμικής λογικής
(Linear Logic).

Στις συνηθισμένες λογικές/γλώσσες προγραμματισμού χρησιμοποιούμε $\phi \vdash \psi$
εννοώντας ότι εφόσον ξέρουμε την εγκυρότητα του $\phi$ μπορούμε να συμπεράνουμε
την εγκυρότητα του $\psi$. Η σχέση αυτή διατηρείτε για πάντα, ανεξαρτήτως της
κατάστασης, μόλις αποδείξουμε ότι $\psi$ είναι αληθές επειδή το $\phi$ είναι
αληθές δεν μπορούμε ξαφνικά να αλλάξουμε γνώμη, $\phi \rightarrow true$ για
πάντα αλλιώς αλλάζει και το αποτέλεσμα του $\psi$. Αυτό έχει ως αποτέλεσμα
ορισμένα συμπεράσματα να είναι πολύ πιθανά, για παράδειγμα έστω $\Alpha \rightarrow \Beta,
\Alpha \rightarrow \Gamma$, τότε ένα ενδεχόμενο συμπέρασμα μπορεί να είναι $\Alpha \vdash \Beta \land \Gamma$.
Το παραπάνω παράδειγμα αν και συμβατό με την πραγματικότητα δεν μοντελοποιεί πλήρως
ένα \textbf{σωστό} πρόγραμμα. Αν το $\Alpha$ ήταν πρόσβαση σε έναν πόρο όπως ένα αρχείο
δεν μπορούμε να το επαναχρησιμοποιήσουμε και για το $\Beta$  και για το $\Gamma$, τι γίνεται
αν ένα από τα δύο απενεργοποιήσει τον πόρο ή στο παράδειγμα μας κλείσει το αρχείο? Χρειαζόμαστε
λοιπόν έναν τρόπο να ελέγχουμε για αυτές τις περιπτώσεις.

Η γραμμική λογική προσθέτει την έννοια της κατανάλωσης. Με $\Alpha \multimap \Beta$
εννοούμε ότι για \textbf{ένα} $\Alpha$ παράγετε \textbf{ένα} $\Beta$. Προσθέτοντας
επίσης έναν εναλλακτικό τελεστή για το "γραμμικό λογικό και" $\otimes$ ισχύει ότι:

\begin{center}
    \begin{math}
    \Alpha \multimap \Beta,\Alpha \multimap \Gamma, \Alpha \nvdash \Beta \otimes \Gamma
    \end{math}
\end{center}

Το $\Alpha$ καταναλώνετε από την πιο αριστερή οντότητα, κατά σύμβαση, το $\Beta$.
Με τον ίδιο συνειρμό ισχύει  $\Alpha \multimap \Beta,\Alpha \multimap \Gamma, \Alpha,\Alpha \vdash \Beta \otimes \Gamma$
καθώς τώρα έχουμε δύο οντότητες του $\Alpha$ η κάθε μια θα καταναλωθεί από το $\Beta$ και $\Gamma$ αντίστοιχα. Το γεγονός
ότι ο μεταγλωττιστής καταλαβαίνει την γραμμική λογική δεν σημαίνει ότι την αντικαθιστά οποιαδήποτε άλλη πλήρως.

Πολλές φορές είναι απαραίτητο να παραχθούν $v\in\mathbb{N} $ αναφορές (references) αμετάβλητης μορφής, δηλαδή ισχύει:

\begin{equation}
\alpha:T \rightarrow \beta: \&T
\end{equation}

Όμως δεν μπορούμε να έχουμε παραπάνω από μια αναφορά αν είναι μεταβλητή, σε αυτήν την περίπτωση καταναλώνετε:

\begin{equation}
\alpha:T \multimap \beta: \& mut T
\end{equation}

Το ίδιο ισχύει για οποιαδήποτε μη-αντιγράψιμη τιμή:

\begin{equation}
\alpha:T \multimap \beta: T
\end{equation}

Ορισμένοι τύποι όμως είναι υπερβολικά πολύ εύκολο να αντιγραφθούν χρησιμοποιόντας κάποιου είδους
\verb|memcpy|, για παράδειγμα \verb|u8|. Αυτοί οι τύπου είναι αντιγράψιμοι.

\begin{equation}
\alpha:T \textup{ is Copy } \rightarrow \beta:T
\end{equation}

Για κατανόηση, στο συντακτικό της Rust:

\begin{lstlisting}[language=Rust]
struct T;

fn reference_infinite_times(a: T) -> T {
    let b = &a;
    let c = &a;
    let d = &a;
    // let e = &mut a; Error, cant consume variable
    // that is referenced in its lifetime
    return a;
}

fn mutate_reference(mut a: &mut T) -> &mut T {
    let b = a;
    // return a; Error, use of consumed variable 'a'
    return b;
}

fn consume_and_return(a: T) -> T {
    let b = a;
    // let c = a; Error, use of consumed variable 'a'
    return b;
}

fn can_copy(mut a: u8) -> u8 {
    let b = a;
    let c = a;
    let d = &mut a;
    let e = &a;

    return a;
}

fn main() {
    let v0: T = T;

    let v1 = reference_infinite_times(v0);
    //infinite_references(var0); Error,
    //use of consumed variable 'v0'

    let mut v2 = consume_and_return(v1);
    //infinite_references(var1); Error,
    //use of consumed variable 'v1'

    let v3 = mutate_reference(&mut v2);
    //v2 is consumed, mutated and returned as v3

    let v = can_copy(42);
}
\end{lstlisting}

Συγκεκριμένα για τον ESP αυτή η μοντελοποίηση είναι πάρα πολύ χρήσιμη για τα
περιφερειακά του μικροελεγκτή. Το αντικείμενο Peripherals περιέχει
καταναλώσιμες τιμές των περιφερειακών και τα συνέπεια δεν μπορούμε να
τα χρησιμοποιήσουμε παραπάνω από μία φορά. Ως αποτέλεσμα αποφεύγουμε
προβλήματα του τύπου να χρησιμοποιήσουμε το ίδιο PIN σαν input και
output ταυτόχρονα το οποίο προφανώς θα κατέστρεφε το πρόγραμμα.

Κατανοώντας τώρα τον βασικό τρόπο λειτουργίας της Rust
μπορούμε να θέσουμε τις πρώτες μας σταθερές.

\begin{lstlisting}[language=Rust]
    //MQTT Peripherals
    let wifi = peripherals.WIFI;

    //HCSR04 Peripherals
    let trig_pin = peripherals.GPIO11;
    let echo_pin = peripherals.GPIO10;
 \end{lstlisting}
 
 \subsection{Δημιουργώντας τα απαραίτητα drivers}

